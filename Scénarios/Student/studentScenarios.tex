\documentclass[a4paper, 11pt]{article}

\usepackage[T1]{fontenc}
\usepackage[utf8]{inputenc}
\usepackage[french]{babel}

\usepackage[most]{tcolorbox}

\title{Scénarios rôle \bf \'Etudiant}
\author{}

\begin{document}
\maketitle

\section*{Scénario 1}
L'étudiant veut proposer un sujet de mémoire. Il s'est déjà mit d'accord au préalable avec un professeur. Il sait également le lieu où il veut réaliser son stage.

\begin{tcolorbox}
    \begin{enumerate}
        \item Se connecter sur la plateforme avec ses identifiants
        \item Sur la page d'accueil, cliquer / sélectionner le cours spécifique aux mémoires
        \item Si la période le permet, l'étudiant peut remplir un formulaire de proposition
        \item Cliquer sur Envoyer
    \end{enumerate}
\end{tcolorbox}

\section*{Scénario 2}
C'est le moment de choisir un sujet de mémoire définitif. L'étudiant veut alors choisir 5 sujets de mémoire (de celui qui lui donne le plus envie à celui qui lui donne le moins envie).

\begin{tcolorbox}
    \begin{enumerate}
        \item Se connecter sur la plateforme avec ses identifiants
        \item Sur la page d'accueil, cliquer / sélectionner le cours spécifique aux mémoires
        \item Si la période est une période de choix de sujets, l'étudiant aura face à lui un tableau avec tous les sujets
        \item Il peut cocher les sujets qu'il veut
        \item Cliquer sur Envoyer
    \end{enumerate}
\end{tcolorbox}

\section*{Scénario 3}
Quelques sujets son déjà publié et il souhaite réserver 1 ou plusieurs sujets.

\begin{tcolorbox}
    \begin{enumerate}
        \item 
    \end{enumerate}
\end{tcolorbox}

\end{document}