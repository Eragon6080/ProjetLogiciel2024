\documentclass[a4paper,11pt, oneside]{article}
%package

\usepackage[utf8]{inputenc}
\usepackage[T1]{fontenc}
\usepackage[french]{babel}
\usepackage{cite}
\usepackage{url}
\usepackage{pdfpages}
\usepackage{soul}
\usepackage{color}
\usepackage{tcolorbox}
\usepackage{graphicx}
\usepackage{geometry}
\usepackage{array}

\geometry{textwidth=16cm, textheight=26cm}

\newcommand{\li}{\newline}

\title{Vie d'un délivrable}
\author{Gaillard Matthys \and Anderson Rosny \and Bouncer Yassine \and Dierick Luis \and Fundu Oliver\and Marchal Tom}



\begin{document}
\maketitle
\section{Introduction}
    \par L'objectif de ce document est de présenter la vie d'un délivrable. Il s'agit de décrire les différentes étapes de sa vie : depuis
    la création du délivrable (création de la période) jusqu'à sa livraison. 
\section{Explication}
    \par Le professeur est le rôle caractérisant le début de la vie d'un délivrable. Donc, en tant que professeur, je 
    souhaiterais que la vie d'un délivrable soit liée à une période d'évaluation. Ainsi, la vie d'un délivrable commence par la création de la période que je souhaiterais définir.
    Quand je crée une période d'évaluation (mémoire ou IDS), je souhaite spécifier et définir combien d'étapes il y aurait, et ce durant la durée de vie de l'évaluation. Je notifierai
    dans la foulée tous les inscrits de l'UE dont je suis responsable qu'une période d'évaluation a commencé. Cette partie se fera plus ou moins manuelle. Chaque étape d'une période est caractérisée par une date de début et de fin.\li

    \par À cette fin, on pourrait utiliser un créneau défini pour une étape et notifier chaque participant du début et de la fin de chaque étape d'une période pour dire qu'il y a un délivrable à rendre si cela est nécessaire.
    Un délivrable est défini par un type de fichier et une consigne définissant le travail à faire pour ce moment-là. Durant ce laps de temps, l'étudiant pourra déposer (modification par la suite possible) son délivrable sur un répertoire adéquat et personnalisé. Pour chaque période et chaque évaluation, 
    une session "fichierDélivrable" est nécessaire. De ce fait, chaque étudiant aura accès à un répertoire personnel pour déposer son délivrable. \li

    \par Sur une page dédiée, j'aimerais que l'étudiant puisse disposer d'une ligne du temps pour avoir un aperçu visuel du temps qu'il lui reste pour la période d'évaluation en cours.\li 

    \par Pour un délivrable, le répertoire de dépôt ne sera disponible que durant la période d'évaluation de l'étape. Afin de paraitre plus clair, les répertoires non disponibles ou plus disponibles seront grisés et injoignables et on ne pourra plus naviguer vers la page de dépôt adéquate.
\end{document}