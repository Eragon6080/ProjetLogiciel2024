\documentclass[a4paper, 11pt]{article}

\usepackage[T1]{fontenc}
\usepackage[utf8]{inputenc}
\usepackage[french]{babel}

\title{Scénarios rôle \bf \'Etudiant}
\author{}

\begin{document}
\maketitle

\section*{Scénario 1}
L'étudiant se connecte sur la plateforme PIMS avec ses identifiants. Il veut pouvoir proposer un sujet de mémoire. Il s'est déjà mit d'accord au préalable avec un professeur. Il sait également le lieu où il veut réaliser son stage.

\section*{Scénario 2}
L'étudiant se connecte sur la plateforme. C'est le moment de choisir un sujet de mémoire définitif. Il peut alors choisir 5 sujets de mémoire (de celui qui lui donne le plus envie à celui qui lui donne le moins envie).

\section*{Scénario 3}
L'étudiant se connecte sur la plateforme. Quelques sujets son déjà publié et il souhaite réserver 1 ou plusieurs sujets.
\end{document}