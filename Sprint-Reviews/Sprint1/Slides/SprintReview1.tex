\documentclass[numbering=fraction]{beamer}

\usepackage[utf8]{inputenc}
\usepackage[T1]{fontenc}
\usepackage[french]{babel}
\usepackage{blindtext}
\usepackage{tikzsymbols}


\usetheme[progressbar=frametitle]{metropolis}

%Define colors
\definecolor{wuppergreen}{RGB}{85, 171, 38}
\definecolor{background}{RGB}{255,255,255}

%Adding logo to title page
\titlegraphic{\raggedleft \includegraphics[width=3cm]{UNamur.png}}

%Adjust color theme
\setbeamercolor{frametitle}{bg=wuppergreen}
\setbeamercolor{title separator}{fg=wuppergreen}
\setbeamercolor{footline}{fg=gray}
\setbeamercolor{progress bar}{fg=black}

%Adding footer
\setbeamertemplate{frame footer}{\insertshortauthor~(\insertshortinstitute)}

%Set parameters for title page
\title{PIMS : Sprint Review 1}
\author[PIMS]{Luis Dierick \and Gaillard Matthys}
\institute{Université de Namur}
\date{\today}

\begin{document}

\begin{frame}[plain]{}
    \maketitle
\end{frame}

\begin{frame}{Table des matières}
    \tableofcontents
\end{frame}

\section{Enquête contextuelle}
\subsection{Interview SIU}

\begin{frame}{SIU}
    \begin{itemize}
        \item Solution sous forme de plugin Moodle
        \begin{itemize}
            \item Incorporation totale sur la plateforme déjà existante
            \item Technologie basée sur php et javascript (solution assez simple à mettre en place)
            \item Back-end à écrire en arrière pour concevoir la logique algorithmique derrière.
        \end{itemize}
    \end{itemize}
\end{frame}

\subsection{Interview vice-doyenne}


\begin{frame}{Vice-doyenne}
\begin{itemize}
    \item Solution avec une plateforme autonome
    \item Assurer un suivi continu
    \item Algorithme d'attribution en langage fonctionnel \\ $\rightarrow$ à transposer et à implémenter dans le système
    \item[$\Rightarrow$] Problème de base de données
\end{itemize}
\end{frame}
\subsection{Document de Vincent Englebert}
\begin{frame}
    \frametitle{Workflow de désiré : Vincent Englebert}
    \begin{enumerate}
        \item Mise à disposition des sujets.
        \item Demande aux étudiants de choisir et de valider un sujet.
        \item Attribution des sujets et validation de la part du professeur référent.
        \item Suivi des étudiants lors des différents délivrables avec envoi de mail si nécessaire.
        \item Archivage des choix et des attributions.
    \end{enumerate}


\end{frame}


\section{Moodle}
\subsection{Moodle : manière de l'utiliser}
\begin{frame}
    \frametitle{Moodle : Idée}
    \begin{itemize}
        \item Plateforme communiquant avec une APi + api + plugin moodle.
        \item 100\% intégré à la plateforme existante.
        \item Utilisation d'un plugin dont la nouvelle plateforme se servirait comme base de communication. Le plugin
        ne servirait qu'à utiliser d'autres services déjà existant dans Moodle
    \end{itemize}   
    \begin{enumerate}
        \item Utilisation systématique de l'api de moodle pour avoir accès aux différentes fonctionnalités.
        \item Utilisation des requêtes https pour communiquer avec la plateforme.
    \end{enumerate}

\end{frame}
\subsection{Plugin existant à utiliser}
\begin{frame}{Liste des plugins}
    \begin{enumerate}
     \item Plugin des mails
     \item Plannification de l'agenda.
     \item Authentification
     \item Accès à une base de données déjà intégrée.
     \item Gestion des cours.
     \item Dépôt de fichiers
    \end{enumerate}
\end{frame}
\subsection{Question : Comment utiliser le plug-in}
\begin{frame}
    \frametitle{Question : Comment utiliser le plug-in}
    \begin{enumerate}
        \item API + plugin moodle + plateforme autonome. L'api servirait de communication entre la plateforme et le plugin.
        \item Plugin moodle + plateforme. La plateforme sert d'administration et le plugin va chercher les informations directement dans la base de données, tables supplémentaires dans la BD déjà existantes? et moodle gère la gestion du reste.
        \begin{enumerate}
            \item Le professeur/responsable se réserve le droit de modifier les sujets s'ils le désirent.
        \end{enumerate}
    \end{enumerate}
\end{frame}

\section{Product backlog}

\begin{frame}{Product backlog actuel}
\centering
\includegraphics[height=0.8\textheight]{backlog.png}
\end{frame}

\section{Plannification Sprint 2}

\begin{frame}{}
    \begin{itemize}
        \item Liste des priorités à définir clairement.
        \item Finaliser le backlog (Poker Planning).
        \item Base de données à part ou modification de celle déjà exitante?
    \end{itemize}
\end{frame}

\end{document}