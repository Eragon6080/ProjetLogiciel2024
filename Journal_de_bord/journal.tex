\documentclass[11pt]{article}
\usepackage[utf8]{inputenc}
\usepackage{geometry}
\geometry{a4paper, margin=1in}
\usepackage{enumitem}
\usepackage{hyperref}
\usepackage{graphicx} 
\usepackage{xcolor} 
\definecolor{navy}{RGB}{35, 215, 240   }
\title{\includegraphics[width=4cm]{images/logo_unamur.png}\\[1cm] % Mettez à jour le chemin ici
Journal de Bord du Projet Agile}
\author{Equipe Pims}
\date{Commencé le: 14-02-2024}

\begin{document}

\maketitle



\maketitle

\newpage
\section*{Réunions et Activités}




\subsection*{{\color{navy}Stand Meeting - 26.02}}
Voici les notes prises lors du stand meeting :

\begin{itemize}
    \item Les premières choses à faire seraient tout ce qui est lié aux opérations CRUD dans le workflow.
    \item La première tâche serait liée à la proposition du sujet, la spécification des sujets mise à vue par les étudiants, et la modification du choix en cours (suppression, modification).
    \item Prendre en compte qu’un cours peut être géré par plusieurs professeurs.
    \item La logique et la validation devraient être réalisées en dehors de Moodle mais le côté visuel serait sur Webcampus. Il faudrait donc lister ce qui est à part et ce qui est incorporé.
    \item À faire : Regarder les différentes composantes existantes et discuter avec les autres intervenants.
    \item Faire un catalogue des différentes fonctionnalités de Webcampus qui seraient intéressantes. Si elles existent on y va, sinon on les redéveloppe.
    \item Il nous manque clairement la distinction entre professeur et superviseur.
    \item On peut aussi clairement distinguer les différents travaux possibles à rendre.
    \item On doit également séparer l’archivage et l’historique. Les personnes ayant accès à ces deux fonctionnalités peuvent différer.
\end{itemize}







\subsection*{{\color{navy}Daily Scrum - 26.02}}
\textbf{Done:}
\begin{itemize}
  \item Opérations CRUD dans le workflow.
  \item Discussion sur la proposition et la spécification des sujets.
\end{itemize}
\textbf{In Progress:}
\begin{itemize}
  \item Modification du choix en cours (suppression, modification).
\end{itemize}
\textbf{TODO:}
\begin{itemize}
  \item Examiner les différentes composantes existantes.
  \item Discuter avec les autres intervenants.
\end{itemize}
\textbf{Problèmes:}
\begin{itemize}
  \item Manque de distinction claire entre professeur et superviseur.
  \item Séparation nécessaire entre l'archivage et l'historique.
\end{itemize}







\subsection*{{\color{navy}Sprint Review - 13.03}}
\textbf{État d'avancement:}
\begin{itemize}
  \item Complétion et clarification des scénarios.
  \item Début de codage basé sur les maquettes Figma.
\end{itemize}
\textbf{Problèmes :}
\begin{itemize}
  \item Nécessité de décider des technologies pour la base de données.
\end{itemize}
\textbf{Suite:}
\begin{itemize}
  \item Planification du prochain sprint en se concentrant sur la liaison entre les écrans.
\end{itemize}





\subsection*{{\color{navy}Sprint Rétrospective - 15.03}}
\textbf{Axes d'amélioration:}
\begin{itemize}
  \item Augmenter le nombre de Daily Scrums pour améliorer la coordination.
\end{itemize}
\textbf{Points positifs:}
\begin{itemize}
  \item Bonne répartition des tâches.
  \item Bonne communication au sein de l'équipe.
\end{itemize}
\textbf{Suivi des rétros précédentes:}
\begin{itemize}
  \item Continuer de préparer les réunions avec le client.
\end{itemize}
\textbf{Points d'actions:}
\begin{itemize}
  \item Remplir régulièrement le journal de bord.
\end{itemize}






\subsection*{{\color{navy}Sprint Planning  - 15.03}}

\textbf{Figma:}
\begin{itemize}
    \item Terminer les mock-ups et leurs transitions pour chaque rôle afin d’avoir une idée plus claire sur comment s’enchainent les différentes pages et identifier les composants communs entre les rôles.
\end{itemize}

\textbf{Implémentation:}
\begin{itemize}
    \item Début de l'implémentation.
    \item Création des différents visuels qui vont être proposés dans l’application.
    \item Création des liens avec la base de données (plus création de la nôtre si besoin pour complémenter celle déjà proposée).
    \item Demander quelles technologies sont déjà utilisées pour gérer les données sur Webcampus afin d’utiliser des technologies compatibles avec celles-ci.
\end{itemize}

\textbf{Résultats du Poker Planning:}
\begin{itemize}
    \item User story 7 $\rightarrow$ 3
    \item User story 2 $\rightarrow$ 5
    \item User story 20 $\rightarrow$ 5
    \item User story 26 $\rightarrow$ 2
    \item User story 8 $\rightarrow$ 21
    \item User story 30 $\rightarrow$ 3
    \item User story 24 $\rightarrow$ 8
    \item User story 29 $\rightarrow$ 2
    \item User story 12 $\rightarrow$ 3
    \item User story 22 $\rightarrow$ 3
    \item User story 21 $\rightarrow$ 3
    \item User story 38 $\rightarrow$ 5
\end{itemize}








\subsection*{{\color{navy}Daily Scrum - 19.03}}
\textbf{Done:}
\begin{itemize}
  \item Design de la page.
  \item Création d'un formulaire de connexion (Django).
\end{itemize}
\textbf{In Progress:}
\begin{itemize}
  \item Adaptation du code Django aux besoins du projet.
  \item Transformation des mockups en pages web.
\end{itemize}
\textbf{TODO:}
\begin{itemize}
  \item Établir une connexion entre le site web et la base de données.
  \item Consulter la Vice-doyenne pour l'algorithme proposé.
\end{itemize}
\textbf{Problèmes:}
\begin{itemize}
  \item Pas d'accès à la base de données actuelle.
\end{itemize}






\subsection*{{\color{navy}Daily Scrum - 22.03}}

\textbf{Done:}
\begin{itemize}
  \item Avancement dans le design du Frontend.
\end{itemize}

\textbf{In Progress:}
\begin{itemize}
  \item -- (Aucune tâche spécifiée comme en cours pour cette session)
\end{itemize}

\textbf{TODO:}
\begin{itemize}
  \item Faire un schéma (entité-relation) pour la future base de données.
  \item Décider si nous devons créer notre propre base de données.
  \item Demander quelles technologies de base de données sont actuellement utilisées au service informatique pour assurer la compatibilité.
\end{itemize}

\textbf{Problèmes:}
\begin{itemize}
  \item Impossible d'avoir accès à la base de données actuelle.
\end{itemize}






\subsection*{{\color{navy}Daily Scrum - 26.03.2024}}

\textbf{Done:}
\begin{itemize}
  \item Le scénario de l’admin a été clairement défini, avec des fonctionnalités pour définir et modifier les rôles, ainsi qu'ajouter ou supprimer des personnes de la base de données.
  \item Les scénarios de l'étudiant et du professeur ont été définis précédemment.
  \item L'implémentation de la base de données est terminée et le schéma de la BD est considéré comme adéquat pour le moment.
  \item Les scénarios sous forme de vidéos ont été réalisés.
  \item La téléversion des fichiers fonctionne correctement.
\end{itemize}

\textbf{In Progress:}
\begin{itemize}
  \item -- 
\end{itemize}

\textbf{TODO:}
\begin{itemize}
  \item Rédiger la version écrite des scénarios.
  \item Continuer l'avancement du site en fonction des scénarios définis.
\end{itemize}

\textbf{Problèmes:}
\begin{itemize}
  \item -- 
\end{itemize}







\subsection*{{\color{navy}Sprint Review - 27.03}}

\textbf{État d'avancement:}
\begin{itemize}
  \item Présentation détaillée des maquettes Figma, avec une mise en avant spécifique pour chaque rôle.
  \item Ajout de déclencheurs aux scénarios suivant le format "trigger-effets".
  \item Isolation de la base de données modèle de la base de données réelle pour réduire les dépendances fortes et faciliter les changements indépendants.
\end{itemize}

\textbf{Problèmes:}
\begin{itemize}
  \item Nécessité d'harmoniser le design avec celui utilisé sur Microsoft Teams.
  \item Amélioration requise pour la fonctionnalité de tri des cours et de la liste des participants.
  \item Réduction du nombre de clics nécessaires pour certaines actions, en particulier pour les tâches répétitives.
\end{itemize}

\textbf{Suite:}
\begin{itemize}
  \item Intégration d'une option pour ajouter plusieurs personnes simultanément.
  \item Ajout d'une fonctionnalité permettant d'épingler des éléments.
  \item Hébergement automatique des fichiers sur OneDrive et réalisation d'une simulation avec un cours fictif.
\end{itemize}

\textbf{Remarques du client:}
\begin{itemize}
  \item Suppression demandée de la demande de confirmation lors de certaines actions.
  \item Proposition d'utiliser des vues pour gérer la séparation entre la base de données modèle et la base de données réelle.
\end{itemize}











\subsection*{{\color{navy}Sprint Rétrospective - 27.03}}

\textbf{Points positifs:}
\begin{itemize}
  \item Amélioration de la méthode Agile appliquée au projet.
  \item Régularité des réunions quotidiennes (dailies), favorisant une bonne coordination.
  \item Cette itération a été menée avec très peu d'erreurs, approchant un sans-faute.
\end{itemize}

\textbf{Axes d'amélioration:}
\begin{itemize}
  \item Mise à jour régulière du Kanban pour refléter l'état actuel des tâches et garantir une vision claire de l'avancement.
  \item Assignation précise et adéquate des tâches dans le product backlog pour éviter toute confusion.
  \item Révision et manipulation du product backlog à chaque réunion quotidienne pour assurer l'alignement de l'équipe et la clarté des objectifs.
\end{itemize}

\textbf{Suivi des rétros précédentes:}
\begin{itemize}
  \item Les actions déterminées lors des rétrospectives précédentes ont été respectées et ont conduit à des améliorations tangibles dans la gestion du projet et la coordination de l'équipe.
\end{itemize}

\textbf{Points d'actions:}
\begin{itemize}
  \item Continuer l'application rigoureuse des méthodes Agile adoptées pour conserver le niveau d'amélioration atteint.
  \item Appliquer les axes d'amélioration identifiés dans cette rétrospective pour le prochain sprint.
  \item Définir des actions concrètes pour chaque point à améliorer et assigner des responsables pour leur mise en œuvre.
\end{itemize}














\subsection*{{\color{navy}Sprint Planning - 29.03}}

\textbf{Poker Planning:}
\begin{itemize}
    \item User story 7 $\rightarrow$ 3
    \item User story 6 $\rightarrow$ 8
    \item User story 19 $\rightarrow$ 3
    \item User story 51 $\rightarrow$ 1
    \item User story 27 $\rightarrow$ 3
    \item User story 49 $\rightarrow$ 2
    \item User story 50 $\rightarrow$ 2
    \item User story 13 $\rightarrow$ 5
    \item User story 22 $\rightarrow$ 3
    \item User story 24 $\rightarrow$ 13
    \item User story 9 $\rightarrow$ 5
\end{itemize}

\textbf{Répartition des tâches:}
\begin{itemize}
    \item Matthys et Luis : Gestion des Rôles par l’admin.
    \item Rosny et Oliver : Timeline des projets (début, délivrables, fin).
    \item Tom et Yassine : Fiche signalétique personne et projet.
\end{itemize}










\subsection*{{\color{navy}Daily Scrum - 04.04.2024}}

\textbf{Done:}
\begin{itemize}
  \item Résolution des problèmes rencontrés avec la fonctionnalité 'récupérer sujet'.
  \item Progression significative dans l'implémentation des user stories assignées.
\end{itemize}

\textbf{In Progress:}
\begin{itemize}
  \item Finalisation des user stories en cours, notamment celles liées à la gestion des rôles par l'admin et la timeline des projets.
\end{itemize}

\textbf{TODO:}
\begin{itemize}
  
  \item Continuer le développement du site en accord avec les scénarios planifiés.
\end{itemize}

\textbf{Problèmes:}
\begin{itemize}
  \item Difficultés à séparer et récupérer les informations de manière efficace pour certaines fonctionnalités.
  \item Accès limité à la base de données nécessitant une évaluation pour une solution potentielle, y compris la création d'une nouvelle base de données si nécessaire.
\end{itemize}






\subsection*{{\color{navy}Daily Scrum - 09.04.2024}}

\textbf{Done:}
\begin{itemize}
  \item Rosny a terminé sa user story.
  \item Luis a terminé sa user story sans rencontrer de problèmes.
\end{itemize}

\textbf{In Progress:}
\begin{itemize}
  \item Yassine terminera sa user story sous peu.
\end{itemize}

\textbf{TODO:}
\begin{itemize}
  \item Attribution des taches 
\end{itemize}

\textbf{Attributions des Tâches:}
\begin{itemize}
  \item Rosny est assigné à la user story 9 - Pouvoir déposer un devoir.
  \item Mathys est assigné à la user story 12.
  \item Oliver est assigné à la user story 74.
  \item Luis est assigné à la user story 65.
\end{itemize}

\textbf{Problèmes:}
\begin{itemize}
  \item Aucun problème majeur rencontré sauf la considération sur la gestion des dates mentionnée par Rosny.
\end{itemize}

\subsection*{{\color{navy}Sprint Review - 11.04}}
\textbf{État d'avancement:}
\begin{itemize}
  \item Au niveau de la gestion des sujets et précisément pour la modification d’un sujet : Ajout d’une fonctionnalité pour undo ( revenir à l’état d’avant )
\end{itemize}
\textbf{Problèmes :}
\begin{itemize}
  \item 
\end{itemize}
\textbf{Suite:}
\begin{itemize}
 \item Faire la chaîne complète pour un sujet donné. Le but ? Lier les morceaux déjà faits
 \item Compléter la définition des modalités d’un délivrable  ( les étudiants qui ont choisi/ sont concerné, type de fichiers, étapes séparées pour chaque délivrable, deadlines ) => logique générale pour le processus des délivrables 
 \item Historique des sujets

\end{itemize}





\subsection*{{\color{navy}Sprint Rétrospective - 12.04}}

\textbf{Points positifs:}
\begin{itemize}
  \item Bonne application de la méthode agile à travers toute l'équipe, mentionnée unanimement comme efficace et bien mise en pratique.
\end{itemize}

\textbf{Axes d'amélioration:}
\begin{itemize}
  \item Dynamiser la sélection des user stories pour s'assurer que les tâches prioritaires sont toujours choisies.
  \item Améliorer la communication en cas de problèmes ou de contraintes de temps pour éviter les retards et les malentendus.
  \item Encourager les membres de l'équipe à demander de l'aide lorsqu'ils en ont besoin pour maintenir la fluidité du travail.
  
\end{itemize}

\textbf{Suivi des rétros précédentes:}
\begin{itemize}
  \item Continuation de l'application rigoureuse des méthodes Agile qui a été fructueuse et a mené à des améliorations tangibles dans la gestion du projet et la coordination de l'équipe.
\end{itemize}

\textbf{Points d'actions:}
\begin{itemize}
  \item Continuer à appliquer et à améliorer l'efficacité des méthodes Agile adoptées.
  \item Mettre en œuvre les axes d'amélioration identifiés durant cette rétrospective pour le prochain sprint.

\end{itemize}



\subsection*{{\color{navy}Sprint Planning - 12.04}}

\textbf{Poker Planning:}
\begin{itemize}
    \item User story 11 $\rightarrow$ 5 (Priorité haute)
    \item User story 14 $\rightarrow$ 5 (Priorité haute)
    \item User story 73 $\rightarrow$ 5 (Priorité haute)
    \item User story 28 $\rightarrow$ 2 (Priorité haute)
    \item User story 5 $\rightarrow$ 1 (Priorité haute)
    \item User story 18 $\rightarrow$ 1 (Priorité haute)
\end{itemize}

\textbf{Répartition des tâches:}
\begin{itemize}
    \item Rosny: Responsable de la User story 73.
    \item Mathys: Responsable des User stories 5 et 18.
    \item Luis et Tom: Responsables de la User story 28.
    \item Oliver: Responsable de la User story 14.
    \item Mathys et Yassin: Responsables de la User story 11.
\end{itemize}




\subsection*{{\color{navy}Daily Scrum - 17.04.2024}}

\textbf{Done:}
\begin{itemize}
  \item Mathys : User story terminée.
  \item Luis : Finalise la partie administration.
\end{itemize}

\textbf{In Progress:}
\begin{itemize}
  \item Luis : Finalise la partie administration.
\end{itemize}

\textbf{TODO:}
\begin{itemize}
  \item Yassine : Résoudre les problèmes rencontrés avec Django, prévu d'être réglé sous peu.
  \item Tom : Continuer à travailler sur les résolutions des problèmes liés à Django.
  \item Oliver : Souci sur Django qui nécessite attention et résolution.
\end{itemize}

\textbf{Problèmes:}
\begin{itemize}
  \item Oliver : Souci sur Django qui nécessite attention et résolution.
  \item Yassine et Tom : Problèmes continus avec Django, bien que des solutions soient en cours d'élaboration.
\end{itemize}







\subsection*{{\color{navy}Daily Scrum - 19.04.2024}}

\textbf{Done:}
\begin{itemize}
  \item Tom a revu l'organisation des pages, avec une nouvelle page d'accueil pour différentes fonctionnalités par rôle.
  \item Yassin et Oliver ont travaillé sur la ligne du temps et l'historique, bien qu'il reste des ajustements à faire.
  \item Matthys Gaillard a terminé ses parties concernant l'attribution des sujets.
  \item Rosny a avancé sur la partie dépôt et l'activation des dépôts conditionnels.
\end{itemize}

\textbf{In Progress:}
\begin{itemize}
  \item Revoir la ligne du temps pour implémenter la gestion via Bootstrap et Django.
  \item Intégration du tri de l'historique par des vues tabulaires.
\end{itemize}

\textbf{TODO:}
\begin{itemize}
  \item Matthys travaillera sur le rôle des superviseurs et ajoutera la possibilité de réserver des sujets.
  \item Rosny apportera des modifications pour permettre de retirer ou modifier un délivrable en temps voulu.
  \item Oliver continuera de gérer la ligne du temps avec des améliorations spécifiques.
  \item Yassine continuera de développer la fonctionnalité d'archivage en utilisant des vues.
  \item Tom ajustera le système de réservation des sujets.
  \item Luis se chargera de récupérer les informations des sujets.
\end{itemize}

\textbf{Problèmes:}
\begin{itemize}
  \item Problèmes d'utilisation de JavaScript pour le tri de l'historique, nécessitant un changement vers des vues tabulaires.
\end{itemize}








\subsection*{{\color{navy}Daily Scrum - 21.04.2024}}

\textbf{Done:}
\begin{itemize}
  \item Oli: User story almost finished.
  \item Luis: US completed.
\end{itemize}

\textbf{In Progress:}
\begin{itemize}
  \item Tom: Task dependent on the archiving part.
\end{itemize}

\textbf{TODO:}
\begin{itemize}
  \item Adjust attributes so that URLs work again.
\end{itemize}

\textbf{Problems:}
\begin{itemize}
  \item -- 
\end{itemize}








\subsection*{{\color{navy}Sprint Review - 24.04}}

\textbf{État d'avancement:}
\begin{itemize}
  \item Pour le rôle admin, archivage a été réalisé qui fait un dump des années précédentes avec une vue dense de toutes les infos liées aux cours, triées par année, cours, professeur/superviseur.
  \item Pour le rôle professeur, ajout de fonctionnalités pour visualiser la ligne du temps pour un prof donné pour un cours et tous ses cours à lui. Date de début ajoutée dans le formulaire de modalités.
\end{itemize}

\textbf{Problèmes:}
\begin{itemize}
  \item Nécessité de continuer à ajuster et affiner l'implémentation de la base de données pour mieux gérer les sujets. Deux types de sujets ont été définis: un sujet pour tous les étudiants et plusieurs sujets pour plusieurs étudiants.
\end{itemize}

\textbf{Suite:}
\begin{itemize}
  \item Continuer le développement et l'intégration des fonctionnalités selon les rôles définis, assurant que toutes les pièces fonctionnent ensemble de manière cohérente.
  \item Compléter la définition des modalités d'un délivrable, en se concentrant sur les détails comme les étudiants concernés, le type de fichiers, et les étapes séparées pour chaque délivrable avec leurs deadlines respectives.
  \item Développer un historique complet des sujets pour faciliter l'archivage et la récupération des informations.
\end{itemize}











\subsection*{{\color{navy}Sprint Rétrospective - 25.04}}

\textbf{Points positifs:}
\begin{itemize}
  \item Bonne continuation de l'utilisation efficace de la méthode Agile à travers toute l'équipe.
  \item Communication constamment améliorée et efficace entre les membres de l'équipe.
\end{itemize}

\textbf{Axes d'amélioration:}
\begin{itemize}
  \item Nécessité de mieux préparer les présentations pour éviter les précipitations de dernière minute et garantir une meilleure qualité.
  \item Mettre à jour régulièrement le Taiga pour refléter l'état actuel des tâches et garantir une vision claire de l'avancement.
\end{itemize}

\textbf{Suivi des rétros précédentes:}
\begin{itemize}
  \item Les actions déterminées lors des rétrospectives précédentes ont été largement respectées, contribuant à des améliorations tangibles dans la gestion du projet et la coordination de l'équipe.
\end{itemize}

\textbf{Points d'actions:}
\begin{itemize}
  \item Continuer à appliquer et à améliorer l'efficacité des méthodes Agile adoptées.
  \item Renforcer la préparation pour les présentations futures pour assurer une meilleure qualité et réduire le stress lié aux délais.
  \item Assurer la mise à jour constante du Taiga pour maintenir tous les membres informés de l'avancement des tâches.
\end{itemize}








\subsection*{{\color{navy}Sprint Planning - 25.04}}

\textbf{Poker Planning:}
\begin{itemize}
    \item User story 107 $\rightarrow$ 3
    \item User story 105 $\rightarrow$ 3
    \item User story 4 $\rightarrow$ 5
    \item User story 109 $\rightarrow$ 3
    \item User story 108 $\rightarrow$ 8
    \item User story 110 $\rightarrow$ 5
    \item User story 104 $\rightarrow$ 5
    \item User story 98 $\rightarrow$ 8
    \item User story 11 $\rightarrow$ 5
    \item User story 84 $\rightarrow$ 5
    \item User story 106 $\rightarrow$ 3
    \item User story 23 $\rightarrow$ 1
\end{itemize}

\textbf{Répartition des tâches:}
\begin{itemize}
    \item Luis: Responsable des User stories 106, 104.
    \item Tom: Responsable de la User story 4.
    \item Mathys: Responsable des User stories 110, 23.
    \item Rosny et Oliver: Responsable de la User story 108.
\end{itemize}

\subsection*{{\color{navy}Daily Scrum - 30.04.2024}}

\textbf{Done:}
\begin{itemize}
  \item Luis : Travail sur l'étape de lien entre l'étape et l'UE, et sur la fonctionnalité permettant à l'admin de choisir un professeur/superviseur pour l'étudiant. Peaufinage en cours.
  \item Mathys : Développement de l'aspect confidentiel du mémoire, avec mise en place de la fonctionnalité permettant à l'étudiant de cocher lui-même. Implémentation des triggers et gestion du nombre maximum par sujet.
\end{itemize}

\textbf{In Progress:}
\begin{itemize}
  \item Tom : Travaille activement sur sa User Story.
\end{itemize}

\textbf{TODO:}
\begin{itemize}
  \item Mettre le site web en ligne.
\end{itemize}

\textbf{Problems:}
\begin{itemize}
  \item --
\end{itemize}









\subsection*{{\color{navy}Daily Scrum - 03.05.2024}}

\textbf{Done:}
\begin{itemize}
  \item Luis : Collage des différentes parties, timeline complète.
\end{itemize}

\textbf{In Progress:}
\begin{itemize}
  \item Yas, Mathys : Mise en ligne et réglage du problème lors de la création d'un nouvel étudiant.
\end{itemize}

\textbf{TODO:}
\begin{itemize}
  \item Rosny : Lier la fonctionnalité de dépôt des délivrables.
\end{itemize}

\textbf{Problems:}
\begin{itemize}
  \item -- 
\end{itemize}











\subsection*{{\color{navy}Daily Scrum - 06.05.2024}}

\textbf{Done:}
\begin{itemize}
  \item Tom : Fonctionnalité permettant de modifier un sujet presque finalisée.
  \item Yassine : Trigger bientôt prêt à être pushé.
\end{itemize}

\textbf{In Progress:}
\begin{itemize}
  \item -- 
\end{itemize}

\textbf{TODO:}
\begin{itemize}
  \item Préparation complète pour la présentation finale (mercredi 06 mai).
\end{itemize}

\textbf{Problems:}
\begin{itemize}
  \item -- 
\end{itemize}




\end{document}