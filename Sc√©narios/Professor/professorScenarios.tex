\documentclass[a4paper, 11pt]{article}

\usepackage[T1]{fontenc}
\usepackage[utf8]{inputenc}
\usepackage[french]{babel}

\usepackage[most]{tcolorbox}


\title{Scénarios rôle \bf Professeur}
\author{}

\begin{document}

\maketitle

\section*{Scénario 1}
Un professeur veut modifier les modalités d'un cours dont il est chargé.

\begin{tcolorbox}
    \begin{enumerate}
        \item Se connecter sur la platerforme avec ses identifiants
        \item Sur la page d'accueil, cliquer / selectionner le cours en question
        \item Une fois sur la page du cours, aller dans l'onglet \og Administration\fg{}
        \item Une fois les changements effectués ne pas oublier d'enregistrer.
    \end{enumerate}
\end{tcolorbox}

\section*{Scénario 2}
Un professeur a une nouvelle idée de sujet pour un mémoire. Il veut donc la rajoutée à sa liste de sujets personnels.

\begin{tcolorbox}
    \begin{enumerate}
        \item Se connecter sur la platerforme avec ses identifiants
        \item Sur la page d'accueil, cliquer / selectionner le cours spécifique aux mémoires
        \item Une fois sur la page du cours, aller dans l'onglet \og Add topic\fg{}
        \item Remplir les différents chants du formulaire et le soumettre.
        \item Redirection vers l'onglet \og My topics\fg{} où on peut voir l'ensemble de ses propres sujets
    \end{enumerate}
\end{tcolorbox}

\section*{Scénario 3}
Un sujet de mémoire d'un professeur doit subir une modification quelconque. Il veut donc pouvoir modifier ce sujet.

\begin{tcolorbox}
    \begin{enumerate}
        \item Se connecter sur la platerforme avec ses identifiants
        \item Sur la page d'accueil, cliquer / selectionner le cours spécifique aux mémoires
        \item Une fois sur la page du cours, aller dans l'onglet \og My topics\fg{}
        \item Ici, il est possible de cliquer sur l'un de ses sujets afin d'en avoir les détails
        \item Sur la page de détails, on modifie ce que l'on souhaite puis on enregistre
        \item Redirection vers l'onglet \og My topics\fg{}
    \end{enumerate}
\end{tcolorbox}



\section*{Scénario 4}
Le choix des sujets a été ouvert depuis un certain temps. Il aimerait avoir une vue d'ensemble sur quels sujets sont prisés et par qui.

\begin{tcolorbox}[title=Si le professeur souhaite voir l'état de ses sujets]
    \begin{enumerate}
        \item Se connecter sur la platerforme avec ses identifiants
        \item Sur la page d'accueil, cliquer / selectionner le cours spécifique aux mémoires
        \item Une fois sur la page du cours, aller dans l'onglet \og My topics\fg{}
        \item Sur cette page, il retrouve un tableau montrant ses sujets ainsi que le nom des étudiants intéressé par celui-ci.
    \end{enumerate}
\end{tcolorbox}

\begin{tcolorbox}[title=Si le professeur est le gérant du cours et souhaite voir l'avancement général des choses]
    \begin{enumerate}
        \item Se connecter sur la platerforme avec ses identifiants
        \item Sur la page d'accueil, cliquer / selectionner le cours spécifique aux mémoires
        \item Une fois sur la page du cours, aller dans l'onglet \og Topics\fg{}
        \item Sur cette page, il retrouve un tableau montrant les sujets, les professeurs en charge des sujets et le nom des étudiants intéressé par ces sujets.
    \end{enumerate}
\end{tcolorbox}



\end{document}